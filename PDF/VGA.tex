\documentclass[10pt,twoside,a4paper]{article}
\usepackage[cm]{fullpage}

%\usepackage{geometry}
%\geometry{a4paper,top=3cm,bottom=3cm,left=3.5cm,right=3.5cm,heightrounded,bindingoffset=5mm}

\usepackage{amsmath}
\usepackage{amsfonts}
\usepackage{amssymb}

%\usepackage{showlabels} % debug, togliere

\usepackage[italian]{babel}
\usepackage[latin1]{inputenc}
%\usepackage[utf8x]{inputenc}
\usepackage{dsfont}
%\usepackage{amsthm}
%\usepackage{enumerate}
%\usepackage{graphicx}
%\usepackage{caption}
%\usepackage{subcaption}
%\usepackage{extarrows}
%\usepackage{mathrsfs}
%\usepackage{braket}
%\usepackage{wrapfig}
%\usepackage{booktabs}
%\usepackage{ifthen}
%\usepackage{tikz}
%\usepackage{pgfplots}
%\usepackage{pgfplotstable}
%\usepackage{verbatim}
%\usepackage{multirow}
%\usepackage{listings}

\usepackage{hyperref}%deve essere l'ultimo package
%http://www.tug.org/applications/hyperref/manual.html
\hypersetup{colorlinks=true,
linkcolor=black}

%\pgfplotsset{compat=1.12}%imposta quale versione di gnuplot usare
%\usetikzlibrary{mindmap,positioning}
%derivate
\newcommand{\de}[2]{\ensuremath{\frac{\mathrm{d} #1}{\mathrm{d} #2}}}
\newcommand{\lde}[2]{\ensuremath{{\mathrm{d} #1}/{\mathrm{d} #2}}}
\newcommand{\dt}[2]{\ensuremath{\frac{\mathrm{d} #1}{\mathrm{d} t}}}
\newcommand{\Dt}[1]{\ensuremath{\frac{\mathrm{D} #1}{\mathrm{D} t}}}
\newcommand{\pde}[2]{\ensuremath{\frac{\partial #1}{\partial #2}}}
\newcommand{\lpde}[2]{\ensuremath{{\partial #1}/{\partial #2}}}
%lettere
\newcommand{\df}{\ensuremath{\mathrm{d}}}
\newcommand{\w}{\ensuremath{\omega}}
\newcommand{\R}{\mathds{R}}
\renewcommand{\O}{\mathcal{O}} %ordine di
%parentesi
\newcommand{\lr}[3]{\ensuremath{\left#1 #3 \right#2}}
\newcommand{\lrt}[1]{\lr{(}{)}{#1}}
\newcommand{\lrq}[1]{\lr{[}{]}{#1}}
\newcommand{\lrg}[1]{\lr{\{}{\}}{#1}}
\newcommand{\norm}[1]{\lr{\|}{\|}{#1}}
\newcommand{\media}[1]{\lr{<}{>}{#1}}

\newcommand{\infint}{\int\limits_{-\infty}^{+\infty}}
\newcommand{\Schrodinger}{Schr\"{o}dinger }
\renewcommand{\o}{\ensuremath{\omega}}
\newcommand{\from}{\leftarrow}

%vettori
\renewcommand{\vec}[1]{\boldsymbol{#1}}

%\numberwithin{equation}{subsection}

%opening
\title{}
\author{Alex Amato & Daniele Rapetti}
\date{}

%\makeindex

\begin{document}
%	\maketitle|
%	\tableofcontents
\section{}
Per controllare un display con una vga serve sapere delle cose

\subsection{Preliminari}
Bisogna conoscere il framerate dello schermo, ovvero la quantit\`a di volte al secondo in cui viene disegnata l'immagine su di esso. Per ogni frame visualizzato lo schermo per un certo periodo rimane nero, bisogna tenere conto di questo vuoto con un parametro detto \textbf{retrace factor}. Infine bisogna aver presente la risoluzione dello schermo (ad esempio $1280\times1024$).

Con questi parametri si calcola il pixelclock, ovvero la frequenza con cui vengono mandate le informazioni allo schermo:
\begin{equation}\label{eq:pixelclock}
PixelClock = \frac{(Horiz\ Res) \times (Vert\ Res) \times (Frame\ Rate)}
{Retrace Factor}
\end{equation}

\subsection{Un frame}
Ora cercher\`o di descrivere come funziona un frame, poi Alex legger\`a e corregger\`a.

\begin{figure}[hbt]
	\centering
\begin{tikzpicture}
\def	\fp	{0.7};
\def	\bp	{1.1};
\def	\s	{0.6};
\def	\d	{2.7};
\def	\h	{0.3};
\def	\hb	{2.7};
\begin{scope}[help lines]%[gray,ultra thin]
\draw (\fp,-2)--++(0,\hb);
\draw (\fp+\s,-2)--++(0,\hb);
\draw (\fp+\s+\bp,-2)--++(0,\hb);
\draw (\fp+\s+\bp+\d,-2)--++(0,\hb);
\draw (\fp+\s+\bp+\d+\fp,-2)--++(0,\hb);
\end{scope}
\begin{scope}[<->]
\draw (\fp,-1.5) --+ (\s,0) node [sloped,midway,above] {\tiny sync};
\draw (\fp+\s,-1.3) --+ (\bp,0) node [sloped,midway,above] {\tiny back} node [sloped,midway,below]{\tiny porch};
\draw (\fp+\s+\bp,-1.5) --+ (\d,0) node [sloped,midway,above] {\tiny display} node [sloped,midway,below]{\tiny time};
\draw (\fp+\s+\bp+\d,-1.3) --+ (\fp,0) node [sloped,midway,above] {\tiny front} node [sloped,midway,below]{\tiny porch};
\end{scope}
\draw (0,0)-|++(\fp+\s,0)-|++(\bp,+\h)-|++(\d,-\h)-|++(\fp+\s+\bp,0) node[above]{D\_enable};
\draw (0,-1+\h)-|++(\fp,-\h)-|++(\s,+\h)-|++(\bp+\d+\fp,-\h)-|++(\s,+\h)-|++(\bp,0)node[below]{SYNC};;
\end{tikzpicture}
\caption{I segnali sincronismo e display}
\end{figure}

Per gestire le tempistiche del display devo rispettare delle precise indicazioni:

Ho un segnale principale che si chiama \textit{HSYNC} che indica allo schermo quando cambiare riga, \`e normalmente alto. Dopo il segnale di \textit{HSYNC} devo aspettare un tempo chiamato \textbf{back porch} dopo il quale c'\`e il \textit{display time} che in impulsi di clock,coincide con il numero di pixel sulla linea, durante il quale mi occupo di inviare allo schermo le informazioni sul colore del singolo pixel, quindi c'\`e un ulteriore tempo morto, detto \textbf{front porch} in cui abbasso l'enable del display e aspetto il seguente segnale di sincronia.

Per il segnale di sincronia verticale il discorso \`e lo stesso, con la differenza che le durate dei vari segmenti sono ``linee orizzontali'' invece che impulsi di clock.



\end{document}